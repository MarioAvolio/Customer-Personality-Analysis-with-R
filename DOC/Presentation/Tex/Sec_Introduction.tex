%---------------------------------------------------------------------------%
\lecture{Research Presentation}{lec_present_intro}
%---------------------------------------------------------------------------%
\section{Introduzione}%
%---------------------------------------------------------------------------%
\begin{frame}[fragile]
    \frametitle{Analisi del Dominio e Obiettivi}
    \begin{enumerate}
        \item Analisi dettagliata dei clienti
        \item Aiutare un'attività commerciale a comprendere meglio i propri compratori
        \item Rendere più semplice la modifica e la scelta dei propri prodotti, in relazione alle esigenze richieste dagli acquirenti
        \item Diverse personalità e comportamenti che gli acquirenti assumono durante il ruolo di potenziali clienti aziendali
        \begin{itemize}
            \item Le aziende non possono adottare lo stesso approccio per ogni tipologia di plausibile compratore
        \end{itemize}
       
        
    \end{enumerate}
\end{frame}
%---------------------------------------------------------------------------%
% \begin{frame}[fragile]
%     \frametitle{Scelte di Design}
%     \begin{enumerate}
%         \item Data
%             \begin{itemize}
%                 \item File che devono essere letti in R per eseguire l’ analisi e che non devono essere modificati.
%                 \item DataSet
%             \end{itemize}
%         \item Script
%         \item Output
%         \begin{itemize}
%                 \item Risultati dei Plots
%             \end{itemize}
%     \end{enumerate}
% \end{frame}
%---------------------------------------------------------------------------%
\begin{frame}[fragile]
\frametitle{Descrizione dei Dati}
\begin{enumerate}
    \item Informazioni Personali
    \begin{itemize}
        \item Grado educativo
        \item Reddito
        \item Numero di figli
        \item Età
    \end{itemize}
    \item Prodotti e Spese
    \begin{itemize}
        \item Spesa totale, negli ultimi due anni, di un prodotto di determinato genere.
    \end{itemize}
    \item Promozioni e offerte
    \begin{itemize}
        \item Offerte accettate delle diverse campagne presenti.
    \end{itemize}
    \item Luoghi e acquisti
    \begin{itemize}
        \item Numero di compere effettuate in un determinato luogo o in un determinato modo.
    \end{itemize}
\end{enumerate}
\end{frame}



